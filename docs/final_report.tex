\documentclass[10pt, letterpaper]{scrartcl}
\usepackage [english]{babel}
\usepackage{enumitem}
\usepackage{geometry}
\usepackage{color}
\usepackage{tikz}
\usepackage{listings}
\usepackage{latexsym}
\usepackage{amsmath}
\usepackage{multirow}
\usetikzlibrary{snakes}
\usetikzlibrary{patterns}
\usepackage[loose]{subfigure}
\usepackage[pdfborder={0 0 0}]{hyperref}


\geometry{margin=2.5cm}

\title{Sasuke - Rootkit Detector group4}
\subtitle{TUM \\Chair of IT Security\\  Rootkit Programming WS2014/15}
\author{Martin Herrman \and Gurusiddesha Chandrasekhara}
\date{\today}

\begin{document}
\maketitle
\tableofcontents
\newpage

\section{Introduction}
This rootkit-detector has been implemented as part of the "Rootkit programming" lab course of the TUM in W2014/14. 

\section{Designing of sasuke}

\section{Detection}

\subsection{Group 1}
\subsection{Group 2}
\subsection{Group 3}
\subsection{Group 5}
\subsection{Group 6}
\subsection{Group 7}

\end{document}
